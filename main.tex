\documentclass[conference]{IEEEtran}
\usepackage{amsmath}
\usepackage{amssymb}
\usepackage{amsfonts}
\usepackage{algorithmic}
\usepackage{graphicx}
\usepackage{textcomp}
\usepackage{xcolor}
\def\BibTeX{{\rm B\kern-.05em{\sc i\kern-.025em b}\kern-.08em
    T\kern-.1667em\lower.7ex\hbox{E}\kern-.125emX}}
\begin{document}

\title{Diseño de un filtro analógico de tipo microstrip en un PCB de FR4 usando Keysight ADS \\
}

\author{\IEEEauthorblockN{1\textsuperscript{st} Erick-Andrés Obregón-Fonseca} \\
\IEEEauthorblockA{\textit{Maestría en Electrónica} \\
\textit{Tecnológico de Costa Rica}\\
Cartago, Costa Rica \\
erickof@ieee.org}
\and
\IEEEauthorblockN{2\textsuperscript{nd} Arturo Córdoba Villalobos} \\
\IEEEauthorblockA{\textit{Maestría en Electrónica} \\
\textit{Tecnológico de Costa Rica}\\
Cartago, Costa Rica \\
arturocv16@gmail.com}
}

\maketitle

\begin{abstract}
Microstrip filters demonstrate remarkable versatility across various disciplines, including communication systems, signal processing, electromagnetic compatibility, wireless technology, and various commercial and industrial applications. For this work, the design of a microwave discrete and microstrip filter is presented following the steps of ADS Circuit Design Cookbook 2.0.

The limitation analysis of the theoretical values against the real world is exhibited with possible solutions that can help to optimize the parameters, like the ADS Swap Parameter analysis. Also, other analyses like the ADS Momentum Simulator can help to see how the model behaves in the presence of connected impedances.
\end{abstract}

\begin{IEEEkeywords}
ADS Keysight, filtro analógico, FR4, microstrip, PCB
\end{IEEEkeywords}

\section{Introducción}
Los filtros analógicos de tipo microstrip han demostrado su utilidad en una variedad de áreas. Se emplean en sistemas de comunicación para aplicaciones de radio frecuencia~\cite{zhang2023multi}, en el procesamiento de señales para la eliminación de ruido o la extracción de bandas de frecuencia específicas~\cite{islam2021spectrum}. Además, se utilizan para asegurar la compatibilidad electromagnética mediante la supresión de interferencia electromagnética y harmónicos~\cite{kumar2023electromagnetic}. Estos filtros son fundamentales en tecnología inalámbrica~\cite{ibrahim2020compact}, así como en aplicaciones comerciales e industriales, incluyendo dispositivos médicos, electrónica automotriz, aeroespacial y sistemas de defensa.


\section{Teoría}
Un filtro se puede descibir como una red reactiva que pasa una banda de frecuencias deseada mientras practicamente rechaza las otras bandas de frecuencias, lo que quiere decir que procesa sañales de forma dependiente de la frecuencia~\cite{zumbahlen2008581}. Una característica importante es la frecuencia de corte ($f_{c}$), la cual es la frecuencia que separa la banda de transmisión de la banda de atenuación. Los filtros pueden clasificarse en pasa altas, pasa bajas, pasa banda y rechaza banda~\cite{keysight_csc2}. \\

El primer paso es seleccionar una aproximación apropiada del prototipo del modelo basado en las especificaciones. El orden se puede calcular:

\subsection*{Aproximación de Butterworth}
\begin{equation}
    L_{A}(\omega^{'}) = 10 log_{10} \left\{1 + \varepsilon \left( \omega^{'} / \omega_{c} \right)^{2N} \right\}
\end{equation}

donde $\varepsilon = \left\{ Antilog_{10} \right( L_{A} / 10 \left) \right\}$ y $L_{A} = 3$db para Butterworth.


\subsection*{Aproximación de Chebyshev}
\begin{equation}
    L_{A}(\omega^{'}) =
    \begin{cases}
        10 log_{10} \left[ 1 + \varepsilon cos^{2} \left( ncos^{-1} \frac{\omega^{'}}{\omega_{c}} \right)^{2N} \right] & \omega^{'} \leq \omega^{'}_{1} \\
        10 log_{10} \left[ 1 + \varepsilon cosh^{2} \left( ncosh^{-1} \frac{\omega^{'}}{\omega_{c}} \right)^{2N} \right] & \omega^{'} \geq \omega^{'}_{1} \\
    \end{cases}
\end{equation}

donde $\omega_{c}$ es la frecuencia angular de corte, $\omega^{'}$ es la frecuencia angular de atenuación, $L_{A}(\omega^{'})$ es la atenuación en $\omega^{'}$, $N$ es el orden del filtro, $\varepsilon = [Antilog_{10} ( L_{Ar} / 10)] - 1$ y $L_{Ar}$ es igual a la ondulación en la banda de paso. \\

Posteriormente, se calculan los valores del filtro dependiendo del tipo de aproximación.


\subsection*{Aproximación de Butterworth}
\begin{subequations}
    \begin{align}
        & g_{0} = 1 \\
        & g_{k} = 2 sin\left\{ \right( 2k - 1 \left) \pi / 2n \right\}~donde~k=1,2,...n \\
        & g_{N + 1} = 1
    \end{align}
\end{subequations}


\subsection*{Aproximación de Chebyshev}
\begin{subequations}
    \begin{align}
        & \beta = ln \left( coth \frac{L_{Ar}}{17.37} \right) \\
        & \gamma = sinh \left( \frac{\beta}{2n} \right) \\
        & a_{k} = sin\left( \frac{2^{k} - \pi}{2n} \right),~k=1,2,3,...,n \\
        & b_{k} = \gamma^{2} + sin^{2}\left( \frac{k \pi}{n} \right),~k=1,2,3,...,n \\
        & g_{1} = \frac{2 a_{1}}{\gamma} \\
        & g_{k} = \frac{4 a_{k - 1} a_{k}}{b_{k - 1} g_{k - 1}},~k=2,3,...,n \\
        & g_{n+1} =
        \begin{cases}
            1 & odd~n \\
            coth^2(\frac{\beta}{4}) & even~n
        \end{cases}
    \end{align}
\end{subequations}

Después de obtener los valores de filtro, se debe hacer la correspondinte transformación de la frecuencia y la impedancia para cumplir con las especificaciones. Los transformaciones se pueden realizar con las siguientes ecuaciones:

\subsection*{Filtro pasa bajas}
\begin{subequations}
    \begin{align}
        & C_{k}^{'} = C_{k} / R_{0} \omega_{c} \\
        & L_{k}^{'} = R_{0} L_{k} / \omega_{c},~donde~R_{0} = 50 \Omega
    \end{align}
\end{subequations}


\subsection*{Filtro pasa bandas}
\begin{subequations}
    \begin{align}
        & L_{1}^{'} = L_{1} Z_{0} / \omega_{0} \Delta \\
        & C_{1}^{'} = \Delta / L_{1} Z_{0} \omega_{0} \\
        & L_{2}^{'} = \Delta Z_{0} / \omega_{0} C_{2} \\
        & C_{2}^{'} = C_{2} / Z_{0} \Delta \omega_{0} \\
        & L_{3}^{'} = L_{3} Z_{0} / \omega_{0} \Delta \\
        & C_{3}^{'} = \Delta / L_{3} Z_{0}  \omega_{0}
    \end{align}
\end{subequations}

donde $\Delta$ es el ancho de banda fraccional $\Delta = (\omega_{2} - \omega_{1}) / \omega_{0}$. \\

\section{Metodología}

Para la creación del filtro, se siguieron los pasos de la guía \textit{Advanced Design System-Circuit Design Cookbook 2.0. Keysight Technologies}~\cite{keysight_csc2}.

\subsection{Creación del layout}

Crear un nuevo layout haciendo click derecho sobre la carpeta del proyecto, seleccionar la opción \texttt{New layout}. \\

Abrir el archivo creado. Haciendo uso de la librería de \texttt{TLines-Microstrip}, utilizar el elemento \texttt{MLIN} para formar las secciones del filtro, esto se puede observar en la Figura \ref{fig:metologia_mlin}. 

\begin{figure}[h!]
    \centering
    \includegraphics[width=8cm]{figures/metodologia/metologia1.png}
    \caption{Parte utilizada para dibujar las secciones del filtro}
    \label{fig:metologia_mlin}
\end{figure}

Colocar el elemento en el área de trabajo demarcado por la zona oscura. Una vez colocado el elemento, darle doble click izquierdo para abrir el panel de parámetros que se puede observar en la Figura \ref{fig:metologia_mlin_dimensiones}.

\begin{figure}[h!]
    \centering
    \includegraphics[width=8cm]{figures/metodologia/metologia2.png}
    \caption{Panel de parámetros del elemento \texttt{MLIN}}
    \label{fig:metologia_mlin_dimensiones}
\end{figure}

Con los pasos mencionados anteriormente, inserte las secciones necesarias utilizando los datos mostrados en la tabla \ref{table:metodologia_dimensiones} hasta formar la estructura mostrada en la figura \ref{fig:metologia_secciones_filtro}, donde el ancho corresponde al parámetro \texttt{W} y el largo a \texttt{L}.

\begin{table}[h!]
\centering
\caption{Dimensiones de las secciones del filtro}
\label{table:metodologia_dimensiones}
\begin{tabular}{|c|c|c|}
\hline
Sección & Ancho {[}mm{]} & Largo {[}mm{]} \\ \hline
TL1     & 2.9            & 4.5            \\ \hline
TL2     & 24.7           & 1.68           \\ \hline
TL3     & 0.66           & 10.145         \\ \hline
TL4     & 24.7           & 4.057          \\ \hline
TL5     & 0.66           & 4.202          \\ \hline
TL6     & 2.9            & 4.5            \\ \hline
\end{tabular}
\end{table}

\begin{figure}[h!]
    \centering
    \includegraphics[width=8cm]{figures/metodologia/metologia3.png}
    \caption{Secciones del filtro}
    \label{fig:metologia_secciones_filtro}
\end{figure}

Una vez completada la estructura del filtro, puede proceder a agregar los pines para la simulación \texttt{EM}, para eso haga click en \texttt{Insert} y seleccione la opción \texttt{Pin}. Coloque uno en la conexión restante de \texttt{TL1} y otro en la conexión restante de \texttt{TL6}.

\subsection{Creación del substrato}

Crear un nuevo substrato haciendo click derecho sobre la carpeta del proyecto, seleccionar la opción \texttt{New substrate}. \\

Modifique el substrato para que se vea como el mostrado en la figura \ref{fig:metologia_substrato}. La capa del dieléctrico debe tener una constante de permeabilidad de 4.6, una tangente de pérdidas de $0.0023$, y un espesor de 1.6mm.
\begin{figure}[h!]
    \centering
    \includegraphics[width=8cm]{figures/metodologia/metologia4.png}
    \caption{Capas del substrato}
    \label{fig:metologia_substrato}
\end{figure}

\subsection{Simulación electromágnetica}

Haga click derecho sobre la carpeta que contiene el layout y seleccione la opción \texttt{New} y posteriormente \texttt{EmSetup}. Seleccione con doble click el archivo creado.
En la pestaña de \texttt{Ports} refresce la información utilizando las 2 flechas verdes en la parte superior izquierda de la ventana. Posteriormente, regrese a la pestaña \texttt{FEM} y seleccione el tipo de simulación \texttt{EM Simulation/Model} utilizando el simulador \texttt{FEM}. En la parte inferior derecha selecciona la opción para generar parámetros y precione el botón \texttt{Simulate}.

\section{Resultados}

Utilizando el simulador FEM de ADS se obtuvieron los parámetros \textit{S} del filtro analógico utilizando micro strips. El resultado obtenido se muestra en la figura     \ref{fig:resultado_parametros_s}. Se puede observar que el filtro paso-bajo tiene una frecuencia de corte de -3.5 dB a 1.67 GHz.

\begin{figure}[!ht]
    \centering
    \includegraphics[width=8cm]{figures/resultado.png}
    \caption{Simulación de parámetros S del filtro}
    \label{fig:resultado_parametros_s}
\end{figure}

\section{Discusión}
La frecuencia de corte de 3dB del filtro diseñado ha experimentado un cambio a 1.68 GHz en lugar de los deseados 2 GHz, debido a las limitaciones de los cálculos teóricos. Estos no consideraron con precisión los efectos de extremo abierto y los cambios repentinos de impedancia en las líneas de transmisión.  \\

Para abordar este inconveniente y restablecer la frecuencia de corte a los 2 GHz deseados, es crucial optimizar las longitudes de las líneas de transmisión. Este proceso puede llevarse a cabo mediante el simulador Momentum en ADS o a través de un barrido paramétrico en las longitudes de las líneas capacitivas e inductivas. \\

El simulador Momentum ofrece una modelización precisa de los efectos de las líneas de transmisión, incluyendo los efectos de extremo abierto y los cambios de impedancia. Además, se puede llevar a cabo una optimización de parámetros mediante un barrido de parámetros en ADS, variando las longitudes de las líneas capacitivas e inductivas dentro de un rango de valores determinado. Esto permitirá analizar el impacto de las diferentes longitudes de línea en la respuesta de frecuencia del filtro. \\

\section{Conclusiones}

El diseño presentado en el \textit{Advanced Design System-Circuit Design Cookbook 2.0. Keysight Technologies} para un filtro analógico de tipo microstrip posee algunas falencias en su guía, ya que presenta información confusa en algunos casos. Un ejemplo de esto es el grosor del material para el dieléctrico FR4 que en una sección indican que es de 4.6 mm pero en otra imagen muestran un grosor de 1.6 mm, y es este último el valor que provee resultados similares a los adjuntos en el manual. Aún así, esta guía presenta una fascinante técnica para la creación de filtros utilizando microstrips, aprovechando las propiedades de impedancia y geometría para buscar las características deseadas. El propio material identifica que el filtro propuesto no cumple con la frecuencia de corte deseada, y proveen una solución útil y enseñan los pasos para realizar un barrido para la optimización del filtro, lo cual expande las oportunidades de diseño que tienen los ingenieros en electrónica para buscar soluciones ópticas que satisfagan los requerimientos del sistema.

\bibliographystyle{IEEEtran}
\bibliography{bibliography/bibliography}

\end{document}
