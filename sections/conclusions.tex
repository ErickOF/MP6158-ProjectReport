\section{Conclusiones}

El diseño presentado en el \textit{Advanced Design System-Circuit Design Cookbook 2.0. Keysight Technologies} para un filtro analógico de tipo microstrip posee algunas falencias en su guía, ya que presenta información confusa en algunos casos. Un ejemplo de esto es el grosor del material para el dieléctrico FR4 que en una sección indican que es de 4.6 mm pero en otra imagen muestran un grosor de 1.6 mm, y es este último el valor que provee resultados similares a los adjuntos en el manual. Aún así, esta guía presenta una fascinante técnica para la creación de filtros utilizando microstrips, aprovechando las propiedades de impedancia y geometría para buscar las características deseadas. El propio material identifica que el filtro propuesto no cumple con la frecuencia de corte deseada, y proveen una solución útil y enseñan los pasos para realizar un barrido para la optimización del filtro, lo cual expande las oportunidades de diseño que tienen los ingenieros en electrónica para buscar soluciones ópticas que satisfagan los requerimientos del sistema.